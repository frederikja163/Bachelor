\section{Introduction}
The formalization of timed regular expressions (TRE) and their equivalence to timed automata (TA) was established in the foundational paper "Timed Regular Expressions" by Eugene Asarin et al.\cite{Eugene2001}.
This paper demonstrated the construction of TAs from TREs, which is a key focus of this paper.
The ability to convert TREs to TAs allows for modeling of temporal systems.

Tools such as MONAA, MONTRE and UPPAAL have been developed to facilitate analysis and verification of temporal systems.
MONAA performs timed pattern matching on timed words by translating TREs into TAs and can visualize these automata using the DOT language for Graphviz.
UPPAAL, a widely recognized tool in the verification of real-time systems, offers robust functionalities for simulating, visualizing, and verifying TAs.

This paper builds upon the methods described by Asarin et al., focusing on the construction and visualization of TAs from TREs.
We aim to enhance the visual representation of these automata through pruning of unnecessary states, edges and clocks, aswell as the application of the Sugiyama framework, thereby improving their readability.
Pruning states, edges and clocks also has the advantage of improving performance, both while generating automata, and during verification using UPPAAL.
By integrating the capabilities of TikZ and UPPAAL, this paper and the resulting product TREAT (Timed Regular Expression Automata Transformation) contribute to the advancement of the field of Timed Regular Expressions and Timed Automata.