\section{Related Work}\label{sec:related work}

% TRE paper: https://www-verimag.imag.fr/~maler/Papers/timed.pdf

% structure
%   meta
%   tre paper
%   monaa
%   uppaal
%   graph paper

This section aims to describe and review the literature that has been foundational for the research and work done in this paper. It contextualises the contents and contributions of this paper in the field of Timed Regular Expressions and graph layout visualisations.

The term \textit{timed regular expression} was formalized in the abtly named paper "Timed Regular Expressions" from 2001, written by Asarin et al.\cite{Eugene2001}. The paper discusses the equivalence between the expressive power of timed automata(TA)\cite{ALUR1994} and timed regular expressions(TRE). To prove this equivalence, they formalise the construction of TAs from TREs, and vice versa. In this paper, only their research on TREs, and how to construct TAs from them, is used.

The aforementioned paper on TREs has been the basis for multiple studies and projects. One such project has resulted in a tool called MONAA. MONAA is a tool used for performing timed pattern matching on timed words \cite{MONAA2017}\cite{MONAAPAPER2018}. When given a TRE, MONAA translates it to an equivalent TA, on which the timed pattern matching is performed. Additionally, since MONAA uses the DOT language for Graphviz \cite{Graphviz} to represent their TA, said automaton can be visualized\footnote{MONAAs automata cannot be directly visualised by Graphviz since they use custom attributes, but the developers of MONAA offer a program for converting to automata recognizable by Graphviz\cite{MONAA2017}.}.
% not sure if i should mention that the tregexs used in monaa, are not as powerful as the ones discusses in the paper?

MONTRE is another tool, that performs timed pattern matching using TREs \cite{MONTRE2016}.

Another aspect of this paper, is finding suitable software for simulating the TA constructed from a given TRE. A tool widely known in the field of verification of real-time systems is UPPAAL. Not only can UPPAAL simulate and visualise TA, it can also validate and verify them\cite{UPPAAL}.

UPPAAL represents their systems (collections of TAs variables etc.) in a handful of formats, the most versatile being XML\cite{UPPAAL}. These XML files contain all the information needed to construct the automata in UPPAAL, including the positions of each individual state. This means that we can use more complex graph layout algorithms to properly visualise TAs. The paper "Visualisation of state machines using the Sugiyama framework" by Viktor Mazetti et al. describes the theory behind and an implementation of the Sugiyama framework, also known as layered graph drawing, which will be discussed later in this paper \cite{Mazetti2012}.