\subsection{Benchmark UPPAAL}\label{sub:benchmark_uppaal}
More states should intuitively mean UPPAAL runs slower, however this might not be the case if these extra states are easily pruneable.
Therefore we tried running the same query twice in UPPAAL on a large timed word.
The timed word used in these examples are 34198 timed characters long, and the last character happens at time 238817.
The data of these benchmarks can be found on \cref{table:uppaal_benchmarks}.

\begin{tabular}{|c|c|c|c|c|}
    \hline
    & States & Edges & Clocks & Time \\
    \hline
    Pruning & 6 & 8 & 3 & 76s \\
    \hline
    No pruning & 10 & 12 & 3 & 103s \\
    \hline
\end{tabular}
\captionof{table}{Table of benchmarks in UPPAAL.}
\label[table]{table:uppaal_benchmarks}

As can be seen on \cref{table:uppaal_benchmarks}, checking is faster with pruning.
This points to UPPAAL not pruning the unnecessary states when checking.
So pruning not only helps with readability, it also helps with performance. 

\subsection{Benchmark TREAT}
\subsubsection{Setup}
Next step is checking if the prunings have an effect on performance on TREAT.
The original implementation pruned steps after generating the entire automaton.
However it might be faster if we prune while generating the automaton.
Therefore we have created a series of regular expressions, and test configurations to put them all against each other and see how each perform.
Each configuration is a mix of pruning before and after generation of the automaton.

\begin{enumerate}
    \item \textbf{N}: This configuration has no pruning.
    \item \textbf{S}: This configuration prunes only states after generation.
    \item \textbf{E}: This configuration prunes everything after generation.
    \item \textbf{SS}: This configuration prunes states during generation and everything after generation.
    \item \textbf{SE}: This configuration prunes states during generation and only states after generation.
    \item \textbf{ES}: This configuration prunes everything during generation and only states after generation.
    \item \textbf{EE}: This configuration prunes everything during generation and everything after generation.
\end{enumerate}

Note: When we say prune during generation, we are talking about pruning the operands of the explosive operators.
This means the left and right operand of an intersection and the single operand of an absorbed iterator.

As for the tests there is a list of regular expressions we have generated to show different aspects of the performance.
Each regular expression is generated with a number, indicating how many times it is repeated.
This is to allow us to graph the results later.

\begin{enumerate}
    \item \textbf{Repeated intersection}: The string "$A[1;1]$" has been repeated n times.
    \item \textbf{Nested absorbed iterator}: The string "$A$" has been nested inside "$(<x>)\oplus$" n times.
    \item \textbf{Nested interval absorbed iterator}: The string "$A[1;1]$" has been nested inside "$(<x>)\oplus$" n times.
    \item \textbf{Interval concat and intersection}: The string "$A[1;10]$" has been repeated n times followed by "$\&A[1;1]$".
    \item \textbf{Concat and intersection}: The string "$A$" has been repeated n times followed by "$\&A$".
    \item \textbf{Concat and absorbed iterator}: The string "$A$" has been repeated n times followed by "$+'$".
\end{enumerate}

Repeated intersection, nested absorbed iterator and nested interval absorbed iterator were all chosen because they would benefit pruning while generating.
Serving as a best case scenario for SS, SE, ES and EE.
The theory being these configurations limit the amount of states used in the explosive operators, therefore making them less explosive.
Ideally it should make them resemble a linear complexity, since the final TAs are of constant length.

The three concat regular expressions where all chosen to benefit no pruning while generating.
Serving as a best case scenario for N, S and E.
The theory being these regular expressions wont benefit from pruning while generating, since a limited or no pruning can be done during generation.

\subsubsection{Results}
The results can be split into 3 main categories.

\begin{enumerate}
    \item Equivalent performance between configurations.
    \item SS, SE, ES and EE configurations are better.
    \item N, S, SS configurations are better.
\end{enumerate}