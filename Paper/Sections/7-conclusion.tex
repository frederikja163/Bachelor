\section{Conclusion}

% Conclude on discussion (results)
%   readability
%   benchmarks
%   future works

In this paper, the program TREAT (Timed Regular Expression to Automaton Transformation), a program to convert TREs to TAs, has been described, and its result have been discussed.
Using TREAT, it is possible to utilize the capabilities of UPPAAL to perform timed pattern matching on timed words, or simply display TAs constructed from TREs in LaTeX using TikZ.

In \cref{subsec:readability}, the implementation of the Sugiyama Framework method has been discussed, which brought clear advantages in the form of visual clarity and better readability, both for UPPAAL and TikZ formats.
Pruning has an arguably more profound effect on readability, since it minimizes the size of the graph.

The benchmarks constructed and performed in \cref{subsec:benchmarks} also show that pruning is not only effective for visual clarity, but also for the general performance of TREAT, both in runtime and in memory allocation.
The benchmarks do, however, show that pruning is not as effective as it could be, primarily do to the fact that, according to our pruning definitions (see \cref{subsec:pruning}), self-looping transitions are never pruned, which snowballs through all other pruning steps.

Ultimately, while potential future features for TREAT are described in \cref{subsec:futureWorks}, the steps described in this paper, have resulted in a program that succesfully enables its users to convert generate TAs from TREs and even perform timed pattern matching on timed words.