\subsection{Pruning}\label{sec:pruning}

Some of the semantics from \cref{subsec:semantics} can lead to automata of an explosive size. An optional pruning step is performed on the automaton, in order to remove transitions, states, and clocks, that do not affect the transition system. This leads to better performance (see \cref{subsec:benchmarks}), and better readability. 
%TODO: Affirm claims made about performance after benchmarks

\subsubsection{Dead State Pruning}\label[subsubsection]{deadStatePruning}
States that are not final states, and that are not the source of any transition, are considered "dead states" according to this pruning rule. Implementing this rule effectively removes any branches of the automaton, that do not end in a final state when applied recursively. Dead state pruning is formally defined in \cref{definition:deadStatePruning}.

Dead states are all the states that have no way of reaching the final state.
These states are pruned by removing all states that have no edges away from them unless they are final states, this is repeated untill no states satisfy this property.

\sembox{
    $\deadstate(A_1)=\left\{\begin{array}{ll}
        A_1 & if \forall q_1\in Q_1:q_1\notin F_1 \transition[_1]\in\Delta_1 \\
        \deadstate(\automaton[][Q][C_1][\Delta][\Sigma_1][s_1][F_1]) & otherwise \\
    \end{array}\right.
    $

    $Q=\{q_1\in Q_1|q_1\notin F_1\wedge\transition[_1]\in\Delta_1\}$

    $\Delta=\{\transition[_1]\in\Delta_1|q_1\in Q\}$
}

\subsubsection{Unreachable State Pruning}\label[subsubsection]{unreachableStatePruning}
States that are not the destination of any transition, are considered unreachable states according to this rule. The initial location is an exception to this rule. When this rule is applied until continuously, it removes any part of the automaton that is disconnected from parts of the automata connected to the initial state. This rule is formally defined in \cref{definition:unreachableStatePruning}

Unreachable states are very similar to dead states.
Except we check for any edges that end at a given state.
If a state is not reachable it is removed.
Repeat until no such states exist.

\sembox{
    $\unreachable(A_1)=\left\{\begin{array}{ll}
        A_1 & if \forall q_1\in Q_1:q_1'\notin F_1\wedge\transition[_1]\in\Delta_1 \\
        \unreachable(\automaton[][Q][C_1][\Delta][\Sigma_1][s_1][F_1]) & otherwise \\
    \end{array}\right.
    $

    $Q=\{q_1'\in Q_1\mid q_1\notin F_1\wedge\transition[_1]\in\Delta_1\}$

    $\Delta=\{\transition[_1]\in\Delta_1\mid q_1'\in Q\}$
}

\subsubsection{Dead Clock Pruning}\label[subsubsection]{deadClockPruning}
During construction and pruning of the automaton, certain clocks may end up not being checked in any transition. This pruning rule defines them as "dead clocks" and prunes them. This rule is formally defined in \cref{definition:deadClockPruning}

% Any clocks that are not used in any intervals can be removed.
% This means removing them both from the clocks set and from any clock resets.
% This shouldn't have a huge impact, but it will allow us to create a more minimal declaration field in Uppaal.
\begin{definition}\label{definition:deadClockPruning}
    Dead clock pruning:
    \vspace{0.5em}

    \sembox{
        $\deadclock\automaton=\automaton[][Q][C'][\Delta'][\Sigma][s][F]$

        \vspace{0.5em}

        $C'=\{c\mid\exists\transition[]\in\Delta:\exists(c',I) \in\phi:c=c'\}$

        $\Delta'=\{\transition[][q][q'][\phi][p\cap C']\mid\transition\in\Delta\}$
    }
\end{definition}

\subsubsection{Dead Edge Pruning}\label[subsubsection]{deadEdgePruning}
Pruning dead edges means pruning all edges that can never be taken because they are overconstrained.

We do this by first taking the intersection of all edges with multiple ranges.
If this intersection has no possible values we know the edge cannot possibly be taken.
Since it cannot be taken we can remove it.

Mathematically this can be described as a function ($\mathbb{E}$) taking in an automaton and returning a new automaton with the dead edges being pruned.

\sembox{
    $\deadedge(A)=\automaton[][Q][C][\Delta'][\Sigma][s][F]$
 
    $\Delta'=\{\transition\in\Delta\mid\phi\neq\emptyset\wedge\emptyset\neq\cap_{i=1}^n\phi_i\}$

}