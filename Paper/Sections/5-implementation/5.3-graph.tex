\subsection{Graph Layout}

% Motivation for implementing algorithm ourselves
%   - alternative: use DOT language for Graphviz. 
%       - too much bloat and better understanding of algorithm if implemened manually
% Sugiyama framework
%   - 4 steps
%       - make graph acyclic
%           - mention trivial solution?
%       - assign states to layers
%       - order states in layers
%       - assign positions

As mentioned earlier, the layout algorithm used in (TODO: Insert name of program) to organize and arrange the states in timed automata, is called the Sugiyama framework, also known as layered graph drawing (TODO: cite sugiyama when \#343 merged). The method comprises four steps that, when combined, results in a layered graph with minimized edge crossings.

% Visualise steps with different automata

\subsubsection{Make automata acyclic}
The first step of the Sugiyama framework is to, temporarily, make the automata acyclic. As described by Mazetti, this is done by reversing a specific list of edges \cite{Mazetti2012}.
Mazetti describes several methods for finding this list of edges, including using a depth first search to visit all states, and any edges that lead to already visited states, should be reversed.
This is the method used by Graphviz' implementation of the Sugiyama framework. % could be irrelevant

This step, however, is trivially completed during the conversion from a timed regular expression to a timed automaton, since the semantics described by Eugene et al. precisely describe when reversible edges are created.
More specifically, in the construction of guaranteed iterator automata, and it's absorbed variant\cite{Eugene2001}.



\subsubsection{Assign states to layers}
\subsubsection{Order states in layers}
\subsubsection{Assign positions}