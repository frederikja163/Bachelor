\subsection{The Product}\label{subsec:theProduct}
The research in this paper has culminated in a software product that allows a user to enter a timed regular expression, and have it converted into a timed automaton.
We have named this piece of software "TREAT". Short for "Timed Regular Expression to Automaton Transformation".

TREAT is accessed through a command line interface. The user can type in their TRE, along with a few options such as adding a timed word thourgh a .csv file, turning off pruning, and silencing any warnings or info messages.
The user can also decide how the TA should be output at this point.

The TRE is parsed into tokens by a (TODO: Type of tokenizer). These tokens are connected as children of one another, to form an abstract syntax tree. 

From the abstract syntax tree, the TA can be created using the rules from Eugene et al., with the modifications described earlier. %TODO: perhaps reference other sections more accurately when paper is closer to completion.
This is done by visiting each node in the tree, and generating states and edges in the TA, according to the rules. Three visitors go through the tree: One checks whether intervals on any given transition is valid. Another goes through the AST, to specifically implement the iterator and the absorbed iterator. The last visitor is responsible for implementing the rules of all other tokens.



\subsubsection{Command Line Interface}

\subsubsection{Tokenizer}

\subsubsection{AST}

\subsubsection{Automaton Generator}

\subsubsection{Pruning}

\subsubsection{Location Assignment}

\subsubsection{Output}

\subsubsection{Timed Words}
