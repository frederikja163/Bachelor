\subsection{Output Formats}\label{subsec:formats}

In order to represent the Timed Automata (TAs) created from Timed Regular Expressions (TREs), we have explored two formats, both described in this subsection.

\subsubsection{UPPAAL}
UPPAAL, mentioned in \cref{sec:related work}, is a state-of-the-art program for representing TAs \cite{UPPAAL}.
When outputting to this format, it is possible to check timed words on the TRE, as further described in \cref{subsec:checking}.
Outputting to UPPAAL is achieved through translating the TAs to an XML file, which can be interpreted by the UPPAAL program.
When outputting to this format, there are several challenges that have to be overcome all of these have to do with checking, so they will also be described in \cref{subsec:checking}.

\subsubsection{TikZ}
TikZ can be used to visualize figures and graphs in LaTeX. TikZ uses some specialized statements for specifying certain parts of a graph \cite{Tikz}.
Outputting graphs in TikZ is trivial when the TA is loaded into memory, as the syntax is simple, and it is only used for visualization.
This output format is useful for getting a simple visualization of one or more TAs, as well as dynamically creating visualizations of TAs for scientific papers written in LaTeX.
