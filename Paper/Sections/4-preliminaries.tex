\section{Preliminaries}\label{sec:preliminaries}

% THE paper: https://www-verimag.imag.fr/~maler/Papers/timed.pdf

% theory of timed automata: https://www.sciencedirect.com/science/article/pii/0304397594900108


% -Outline automata
% -Explain timed symbols
% -Introduce THE paper


The research in this paper is based on the research done in the paper "Timed Regular Expressions" by Asarin et al.\cite{Eugene2001}. The Paper describes the equivalence between timed regular expressions and timed automata. Their methods of constructing a timed automaton from a timed regular expression, are used and expanded upon throughout this paper. It should be noted, that Asarin et al.'s definition of timed regular expressions is also used in this paper.
An automaton can be defined as a set of states, and a set of transitions between those states. This can be visualized as a directed graph of nodes with edges connecting them.
A timed automaton is a type of automaton that can analyse events happening one after the other, after or within, a specified time, and can be formally defined as a quintuple\cite{ALUR1994}. See \cref{definition:automatonDefinition}.
\begin{definition}\label{definition:automatonDefinition}
    Automaton:
    $$\automaton{}{}{}{}{}{}$$

    $Q$: Finite set of states

    $C$: Finite set of clocks
    
    $\Delta$:Transition relation
    
    $\Sigma$: Alphabet
    
    $s\in Q$:Initial state
    
    $F\subseteq Q$:Accepting states
\end{definition}\cite*{Eugene2001}

Clocks are variables that track time. All clocks count up simultaneously, but may be individually reset on transitions.

Transition relations are the transitions between states, and may be formally described with the following tuple. See \cref{definition:transition}.

\begin{definition}\label{definition:transition}
    Transition relation:
    $$\Sigma_\epsilon=\Sigma\cup{\epsilon}$$
    $$\Delta\subseteq Q\times\phi\times C\times\Sigma_\epsilon\times Q$$
The tuples have the form $\transition$

$q$: The starting state of the transition

$\phi$: A set of boolean formulas written like $x\in I$

$p$:Clocks to be reset when the transition is taken

$a$: The symbol of the transition

$q'$: The ending state of the transition
\end{definition}\cite*{Eugene2001}

The alphabet is the set of symbols the automaton can read.

The initial state is the unique state from which the automaton starts, and is entered at time zero.

The final states are a finite set of states. If any of these states are reachable, given a timed word, that timed word is accepted.

The words describe sets of events that happen one after the other. A timed automaton is able to analyse timed words, where each event has an associated time, to determine whether the word is accepted by the automaton.

\begin{definition}\label{definition:TimedWord}
    Timed words:
    $$\mathbb{R}\times\Sigma$$
\end{definition}

We have developed a concrete syntax, with an accompanying precedence, that can be found in \cref{sec:Syntax} and \cref{sec:Precedence}.