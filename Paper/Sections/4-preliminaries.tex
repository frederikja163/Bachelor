\section{Preliminaries}

% THE paper: https://www-verimag.imag.fr/~maler/Papers/timed.pdf

% theory of timed automata: https://www.sciencedirect.com/science/article/pii/0304397594900108


% -Outline automata
% -Explain timed symbols
% -Introduce THE paper


The research in this paper is based on the research of Asarin, Caspi, and Maler \cite{Eugene2001} on timed regular expressions, and their equivalence to timed automata. Their methods of constructing timed automata from a timed regex, are used and expanded upon throughout this paper.
An automata can be defined as a set of states, and a set of transitions between those states. This can be visualized as a directed graph of nodes with edges connecting them.
A timed automata\cite{ALUR1994183} is a type of automata that can analyse events happening one after the other, after or within, a specified time, and can be formally defined as a quintouple:

\begin{definition}\label{definition:automatonDefinition}
    \cite*{Eugene2001}
    Automaton:
    $$\automaton{}{}{}{}{}{}$$

    $Q$: Finite set of states

    $C$: Finite set of clocks
    
    $\Delta$:Transition relation
    
    $\Sigma$: Alphabet
    
    $s\in Q$:Initial state
    
    $F\subseteq Q$:Accepting states
\end{definition}\\

Clocks are variables that track time. All clocks count up simultaneously, but may be individually reset when a specific transition is taken.

Transition relations are the transitions between states, and may be formally described with the following tuple:

\begin{definition}\label{definition:transition}
    Transition relation:
    $$\Sigma_\epsilon=\Sigma\cup{\epsilon}$$
    $$\Delta\subseteq Q\times\phi\times C\times\Sigma_\epsilon\times Q$$
The tuples have the form $\transition$

$q$: The starting state of the transition

$\phi$: A set of boolean formulas written like $x\in I$

$p$:Clocks to be reset when the transition is taken

$a$: The symbol of the transition

$q'$: The ending state of the transition
\end{definition}\cite*{Eugene2001}\\

The alphabet is the set of symbols that are accepted by the automata. The symbols name the events that are to be analysed.

The initial state is the unique state from which the simulation starts, and is entered at time zero.

The accepting states are a set of states that the simulation must be able to reach for a word to be accepted.\\

The words describe sets of events that happen one after the other. A timed automata is able to analyse timed words, where each event has an associated time, to determine whether the word is accepted by the automata.


