When two states are equivalent in all attributes we can merge them into one state.
To consider if two states are equivalent, neither can be an initial state, and both have to either be a final state or neither can be a final state.
Furthermore all edges need to have an equivalent edge between both states.
Equivalent edges need to both go to the same state, have the same symbol, same clock reset and same clock intervals.

If two states are equivalent in this manner, we can merge them by making all edges that go to either of the two states instead go to a new state.
All edges from either one of the states now have to come from the new state aswell.

\sembox{
    $\mathbb{E}(\Delta, s, F, q_a, q_b)=(q_a\neq q_b)\wedge(q_a\in F=q_b\in F)\wedge (q_a\neq s\wedge q_b\neq s)\wedge\forall\transition[][q_a]\in\Delta:\exists\transition[][q_b]\in\Delta$

    $\merge(A_1)=
    \left\{\begin{array}{ll}
        A_1 & if \exists q_a\in Q_1:\exists q_b\in Q_1:\mathbb{E}(\Delta_1,s_1,F_1,q_a,q_b)\\
        \automaton & otherwise\\
    \end{array}\right.
    $

    $Q_{unmerged}=\{q_a\in Q_a\mid\neg\exists q_b\in Q_1:\mathbb{E}(\Delta_1,s_1,F_1,q_a,q_b)\}$

    $Q_{merged}=\{\{q_a,q_b\}\mid q_a\in Q_1\wedge q_b\in Q_1\wedge\mathbb{E}(\Delta_1,s_1,F_1,q_a,q_b)\}$

    $Q=Q_{unmerged}\wedge Q_{merged}$


}