Pruning dead edges means pruning all edges that can never be taken because they are overconstrained.

We do this by first taking the intersection of all edges with multiple ranges.
If this intersection has no possible values we know the edge cannot possibly be taken.
Since it cannot be taken we can remove it.

Mathematically this can be described as a function ($\mathbb{E}$) taking in an automaton and returning a new automaton with the dead edges being pruned.

\sembox{
    $\deadedge(A_1)=\automaton[][Q_1][C_1][\Delta][\Sigma][s_1][F_1]$
 
    $\Delta=\{\transition|\transition\in\Delta_1\wedge\emptyset\neq\cap_{i=1}^n\Phi_i\}$

}